A Magnitude Aparente expressa o brilho aparente de um astro. Hiparcos, por volta de 150 A.C, dividiu as estrelas, visíveis a olho nu, de acordo com seu brilho aparente. Ele alocou às estrelas mais brilhantes do céu uma magnitude m=1, e às menos brilhantes m=6.

Pogson, no século XIX, percebeu que o sistema proposto por Hiparco seguia uma escala logarítimica (relação com a biofísica do olho humano). Uma diferença de 5 magnitudes correspondia a um fator de exatamente 100 vezes em fluxo. Então temos: 

$$m_{1} - m_{2} = K log \left(\frac{F_{1}}{F_{2}} \right) \Rightarrow 1 - 6 = K log \left(\frac{F_{1}}{F_{2}} \right) \Rightarrow -5 = K log(100) \Rightarrow K = -2,5$$

Com o $K = -2,5$ em mãos, podemos voltar para a equação inicial.

$$m_{1} - m_{2} = -2,5 log \left(\frac{F_{1}}{F_{2}} \right) \Rightarrow m_{1} = -2,5 log \left(\frac{F_{1}}{F_{2}} \right) + m_{2}$$

Podemos definir, matematicamente, a \textit{magnitude aparente} $m$ como:

$$m = -2,5 \ log(F) + cte$$

Como o fluxo $F$ depende do comprimento de onda então $m$ também depende.