A \textit{magnitude bolométrica} corresponde ao fluxo em todos os comprimentos de onda.

Ora, é sabido que:
$$F_{\nu} = \int I_{\nu} cos \theta d\omega \Rightarrow F = \lim_{H \to \infty} \int^{H}_{0} F_{\nu} d\nu$$

Ou seja, o fluxo total $F$ é a soma contínua dos "fluxos parciais" $F_{\nu}$ onde $d\nu$ representa um intervalo infinitesimal de frequência. Então temos:

$$F = \frac{L}{4 \pi R^{2}} \Rightarrow L = \lim_{H \to \infty} 4 \pi R^{2} \int^{H}_{0} F_{\nu} d\nu = 4 \pi R^{2} F_{bol}$$

Como a atmosfera impede a passagem de alguns intervalos espectrais, determina-se a magnitude bolométrica utilizando a magnitude visual e aplicando uma certa correção (C.B).

$$m_{bol} = m_{V} - C.B$$

C.B, a correção bolométrica, tem um valor próximo de zero para estrelas parecidas com o Sol e valores maiores para estrelas mais quentes ou mais frias que o Sol.

Se fizermos a diferença entre a magnitude bolométrica entre duas estrelas temos:

$$m_{bol1} - m_{bol2} = m_{V1} - m_{V2} = -2,5 \ log \left( \frac{\frac{L_{1}}{4 \pi d_{1}^{2}}}{\frac{L_{2}}{4 \pi d_{2}^{2}}} \right)$$

 Assumindo distâncias iguais de 10pc:
 
$$m_{bol1} - m_{bol2} = m_{V1} - m_{V2} = -2,5 \ log \left(  \frac{L_{1}}{L_{2}}\right)$$

Assumindo que a estrela 2 seja o Sol, temos:

$$M_{s \ bol} = 4,72 - 2,5 \ log \left( \frac{L}{L_{\odot}} \right)$$

Agora falando um pouco mais sobre a \textbf{correção bolométrica}, analisando as equações expostas aqui é possível defini-la como uma correção que converte a magnitude aparente em magnitude bolométrica.

